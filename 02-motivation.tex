\section{System Performance Metrics}

Computer performance is dependent on a variety of factors, ranging
from very low level features like the latency of individual functional units, bypass networks,
and out-of-order hardware, up through very high level features such as
operating systems, disk I/O, and network latency.  In this paper we
focus on the performance of the LLC.  
The quality of an LLC is typically gauged by two factors, the improvement in
performance, measured in Instructions Per Clock (IPC),
that it affords to the processor it backs, and the reduction of Misses Per
1000 Instructions (MPKI), which equates to a reduction in the
number of long latency DRAM main memory accesses.

\subsection{IPC}

The number of instructions a CPU can complete in a single clock cycle
can be equated with its absolute performance.  If a processor can
complete more instructions in a given clock cycle than another,
its performance is better.  Hardware caches play a critical role in
boosting this number.  The closer that data sits to the functional
units, the higher performance can be.  In a typical three level cache
hierarchy, it can take 10x longer to access the third level of
cache than the first, and another 10x longer to access main memory.
Finding data as close to the processor as possible is critical for
high performance.

\subsection{MPKI}

One of the LLC's main jobs is to redue the number of DRAM accesses that are
performed.  Each DRAM read access includes occupying a memory
controller's read buffer, waiting for this access's turn, and then
sending that read request across long wires to distant DRAM chips,
activating those DRAM chips, and then finally sending the data back to
the memory controller over several more cycles.  This description of
events omits what happens to the data after it gets back to the memory
controller, and also ignores the consequences of needing to write
dirty data from the LLC back to the cache.  There is a lot of work
that has to be done in the event of an LLC miss, so reducing the MPKI
of a cache is a good goal, making MPKI a popular and important metric to consider.

%\subsection{More of the Story}
% This needs work
%
%Measuring IPC and MPKI can tell you about how well your LLC is
%performing, but they don't tell you about what's going on inside the
%LLC to provide the performance.  Looking at IPC alone does not tell
%you anything about where to attribute the increase in performance.
%Similarly, looking at MPKI alone does not tell you about the
%underlying memory access patterns that caused that MPKI.  In this
%paper, we look at the Time to Recache (TTR) metric in order to get a
%more complete view of the behavior of an LLC.
