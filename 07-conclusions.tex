\section{Conclusions}

We have presented a novel cache performance metric we call 
Time-To-Recache (TTR) and have demonstrated how it can be used to 
analyze the cache behavior of a given system. 
While we have not shown a quantitative measure based on the TTR plots 
that can be used in performance comparisons, we believe there are other
ways that TTR can benefit computer architects. 
For example, TTR can be used to discover reuse intervals of a given
program by inspecting the periods where TTR is hot.
Additionally, by comparing different cache sizes a designer can see
how the cache size affects the ability of the cache to work effectively.

We anticipate that TTR can be used for a variety of applications from cache
design for computer architects, through low-level programmers laying out 
program memory to large distributed data center memory caches managing
many terabytes of data.
Each type of design will use TTR in a different way, but this new method 
for visualizing data can be used to great effect and success.
