\section{Results}

In this section we show the results of TTR tracking for our different
benchmarks and cache management policies.  We also include IPC and
MPKI numbers for comparison.  As stated
earlier, the purpose of this paper is not to showcase the relative
performance of one caching policy over another, but is rather to show
the effectiveness of the TTR visualization method at giving some
additional insight into system performance.
Furthermore, no attempt was made to ensure that the simulation windows
we used in gathering these statistics is necessarily representative of
program execution overall.  For this reason, our performance numbers
for the cache policies considered here may differ from previously published
results for those cache policies.  Our goal is merely to show that TTR
visualization is a useful way to explain the performance we observed.

Our results are presented by showing the TTR graphs for LRU, NRU,
SRRIP, and DRRIP for each of the benchmarks, using both 4~MB and
8~MB.  The third graph shows the IPC and MPKI for each of those runs,
again separated by the two different cache sizes.  We have organized
the figures this way to make it easier to compare the shapes of the
TTR visualizations with the observed IPCs and MPKIs for each
benchmark.  When looking at the TTR visualizations for a given
benchmark and a given cache size, the scale for all four component
graphs (LRU, NRU, SRRIP, and DRRIP) is the same in both X and Y
directions, so they are all directly comparable.  When moving between
the 4~MB and 8~MB visualizatoins, the scales may be different.

% This is the format for discussing in the paper

\begin{figure*}
  \centerline{
  \psfig{figure=figures/ep_recache4MB.eps,width=\columnwidth,height=0.18\paperheight}
  \psfig{figure=figures/ep_recache8MB.eps,width=\columnwidth,height=0.18\paperheight}
  }
\caption{Time to Recache}
\label{Fig:performance:ep}
\end{figure*}
\begin{figure*}
  \centerline{
  \psfig{figure=figures/lu_recache4MB.eps,width=\columnwidth,height=0.18\paperheight}
  \psfig{figure=figures/lu_recache8MB.eps,width=\columnwidth,height=0.18\paperheight}
  }
\caption{Time to Recache}
\label{Fig:performance:lu}
\end{figure*}
\begin{figure*}
  \centerline{
  \psfig{figure=figures/mg_recache4MB.eps,width=\columnwidth,height=0.18\paperheight}
  \psfig{figure=figures/mg_recache8MB.eps,width=\columnwidth,height=0.18\paperheight}
  }
\caption{Time to Recache}
\label{Fig:performance:mg}
\end{figure*}

% This is the format for including at the end
\begin{figure*}
\centerline{
  \psfig{figure=figures/sp_recache4MB.eps,width=\columnwidth,height=0.18\paperheight}
  \psfig{figure=figures/sp_recache8MB.eps,width=\columnwidth,height=0.18\paperheight}
}
\centerline{
  \psfig{figure=figures/astar_recache4MB.eps,width=\columnwidth,height=0.18\paperheight}
  \psfig{figure=figures/astar_recache8MB.eps,width=\columnwidth,height=0.18\paperheight}
}
\centerline{
  \psfig{figure=figures/bwaves_recache4MB.eps,width=\columnwidth,height=0.18\paperheight}
  \psfig{figure=figures/bwaves_recache8MB.eps,width=\columnwidth,height=0.18\paperheight}
}
\centerline{
  \psfig{figure=figures/dealII_recache4MB.eps,width=\columnwidth,height=0.18\paperheight}
  \psfig{figure=figures/dealII_recache8MB.eps,width=\columnwidth,height=0.18\paperheight}
}
\caption{Time to Recache}
\label{Fig:performance2}
\end{figure*}

\begin{figure*}
%\hspace{1.6in}
\centerline{
  \psfig{figure=figures/ua_recache4MB.eps,width=\columnwidth,height=0.18\paperheight}
  \psfig{figure=figures/ua_recache8MB.eps,width=\columnwidth,height=0.18\paperheight}
}
\centerline{
  \psfig{figure=figures/bzip2_recache4MB.eps,width=\columnwidth,height=0.18\paperheight}
  \psfig{figure=figures/bzip2_recache8MB.eps,width=\columnwidth,height=0.18\paperheight}
}
\centerline{
  \psfig{figure=figures/GemsFDTD_recache4MB.eps,width=\columnwidth,height=0.18\paperheight}
  \psfig{figure=figures/GemsFDTD_recache8MB.eps,width=\columnwidth,height=0.18\paperheight}
}
\centerline{
  \psfig{figure=figures/gobmk_recache4MB.eps,width=\columnwidth,height=0.18\paperheight}
  \psfig{figure=figures/gobmk_recache8MB.eps,width=\columnwidth,height=0.18\paperheight}
}
\caption{Time to Recache}
\label{Fig:performance3}
\end{figure*}

\begin{figure*}
%\hspace{1.6in}
\centerline{
  \psfig{figure=figures/gromacs_recache4MB.eps,width=\columnwidth,height=0.18\paperheight}
  \psfig{figure=figures/gromacs_recache8MB.eps,width=\columnwidth,height=0.18\paperheight}
}
\centerline{
  \psfig{figure=figures/h264ref_recache4MB.eps,width=\columnwidth,height=0.18\paperheight}
  \psfig{figure=figures/h264ref_recache8MB.eps,width=\columnwidth,height=0.18\paperheight}
}
\centerline{
  \psfig{figure=figures/hmmer_recache4MB.eps,width=\columnwidth,height=0.18\paperheight}
  \psfig{figure=figures/hmmer_recache8MB.eps,width=\columnwidth,height=0.18\paperheight}
}
\centerline{
  \psfig{figure=figures/leslie3d_recache4MB.eps,width=\columnwidth,height=0.18\paperheight}
  \psfig{figure=figures/leslie3d_recache8MB.eps,width=\columnwidth,height=0.18\paperheight}
}
\caption{Time to Recache}
\label{Fig:performance4}
\end{figure*}
