\section{Results}

In this section we show the results of TTR tracking for our different
benchmarks and cache management policies.  
% We also include IPC and MPKI numbers for comparison.  
As stated earlier, the purpose of this paper is not to showcase the relative
performance of one caching policy over another, but is rather to show
the effectiveness of the TTR visualization method at giving
insight into system performance based on cache behavior.
Furthermore, no attempt was made to ensure that the simulation windows
we used in gathering these statistics is necessarily representative of
program execution overall.  
For this reason, the performance numbers for the cache policies considered 
here may differ from previously published results for those cache policies.
Our goal is merely to show that TTR
visualization is a useful way to explain the performance we observed.

% This is the format for discussing in the paper

\begin{figure*}
  \centerline{
  \psfig{figure=figures/ep_recache4MB.eps,width=\columnwidth,height=0.18\paperheight}
  \psfig{figure=figures/ep_recache8MB.eps,width=\columnwidth,height=0.18\paperheight}
  }
% \caption{Time to Recache}
% \label{Fig:performance:ep}
% \end{figure*}
% \begin{figure*}
  \centerline{
  \psfig{figure=figures/lu_recache4MB.eps,width=\columnwidth,height=0.18\paperheight}
  \psfig{figure=figures/lu_recache8MB.eps,width=\columnwidth,height=0.18\paperheight}
  }
% \caption{Time to Recache}
% \label{Fig:performance:lu}
% \end{figure*}
% \begin{figure*}
  \centerline{
  \psfig{figure=figures/mg_recache4MB.eps,width=\columnwidth,height=0.18\paperheight}
  \psfig{figure=figures/mg_recache8MB.eps,width=\columnwidth,height=0.18\paperheight}
  }
\caption{Time to Recache}
\label{Fig:performance:mg}
\end{figure*}

Our results are presented by showing the TTR graphs for LRU, NRU,
SRRIP, and DRRIP for each of the benchmarks, using both 4~MB and
8~MB.  
% The third graph shows the IPC and MPKI for each of those runs,
% again separated by the two different cache sizes.  We have organized
% the figures this way to make it easier to compare the shapes of the
% TTR visualizations with the observed IPCs and MPKIs for each
% benchmark.  
When looking at the TTR visualizations for a given
benchmark and a given cache size, the scale for all four component
graphs (LRU, NRU, SRRIP, and DRRIP) is the same in both X and Y
directions, so they are all directly comparable.  When moving between
the 4~MB and 8~MB visualizatoins, the scales may be different.
Figure~\ref{Fig:performance:mg} shows some of the most interesting 
results, and Figures~\ref{Fig:performance2}-\ref{Fig:performance4} 
can be found at the end of the paper.

